\documentclass[a4paper]{article}
\usepackage[intlimits]{amsmath}
\usepackage[utf8]{inputenc}
\usepackage[margin=0.8in]{geometry}

\begin{document}

\centerline{\sc \large 3B.18: Teslatrasslet}
\vspace{.5pc}
\centerline{\sc Snurrig bana i magnetfält}
\vspace{2pc}

Uppgiften kan delas upp i tre delar. Del 1 går ut på att visa att
fältkomponenterna av en vektor 
$$
B(r, 0, z) = \frac{\mu_r\mu_0}{4\pi} \int_{-\pi}^\pi
\frac{\bf{i}(\varphi)\times\bf{s}(\varphi)}{s^3}ad\varphi
$$
med $\mu_0 = 4\pi \cdot 10^{-7}$,\ $\mathbf{i}(\varphi) = I_0(-sin\varphi,
cos\varphi, 0)$,\ $\mathbf{s}(\varphi) = \mathbf{p} - \mathbf{q} = (r -
acos\varphi, -asin\varphi, z)$ och $s = \|p - q\|_2 = \sqrt{(r - acos\varphi)^2
+ a^2sin^2\varphi + z^2}$ blir
$$
B_x = C\int_{-\pi}^\pi \frac{zcos\varphi}{s^3}d\varphi, \
B_y = 0,\
B_z = C\int_{-\pi}^\pi \frac{a - rcos\varphi}{s^3}d\varphi 
$$
där $C = \mu_rI_0a \cdot 10^{-7}$.

\vspace{1pc}
Det går att lösa enkelt utan nummeriska metoder. 
\begin{eqnarray*}
  B(r, 0, z) & = \frac{\mu_r\mu_0}{4\pi} \int_{-\pi}^\pi
  \frac{\bf{i}(\varphi)\times\bf{s}(\varphi)}{s^3}ad\varphi \\
  & = \frac{a\mu_r\mu_0}{4\pi} \int_{-\pi}^\pi
  \frac{\bf{i}(\varphi)\times\bf{s}(\varphi)}{s^3}d\varphi \\
  & = \frac{4\pi\mu_ra \cdot 10^{-7}}{4\pi} \int_{-\pi}^\pi 
  \frac{\bf{i}(\varphi)\times\bf{s}(\varphi)}{s^3}d\varphi \\
  & = \mu_ra \cdot 10^{-7} \int_{-\pi}^\pi
  \frac{\bf{i}(\varphi)\times\bf{s}(\varphi)}{s^3}d\varphi \\
  & = \mu_ra \cdot 10^{-7} \int_{-\pi}^\pi
  \frac{I_0(-sin\varphi, cos\varphi, 0)\times(r - acos\varphi, -asin\varphi,
  z)}{s^3}d\varphi \\
  & = \mu_rI_0a \cdot 10^{-7} \int_{-\pi}^\pi
  \frac{(-sin\varphi, cos\varphi, 0)\times(r - acos\varphi, -asin\varphi,
  z)}{s^3}d\varphi \\
  & = C \int_{-\pi}^\pi
  \frac{(-sin\varphi, cos\varphi, 0)\times(r - acos\varphi, -asin\varphi,
  z)}{s^3}d\varphi \\
  & = C \int_{-\pi}^\pi
  \frac{(zcos\varphi - 0, 0 - (-zsin\varphi), asin^2\varphi - cos\varphi(r -
  acos\varphi))}{s^3}d\varphi \\
  & = C \int_{-\pi}^\pi
  \frac{(zcos\varphi, zsin\varphi, asin^2\varphi + acos^2\varphi - rcos\varphi)}{s^3}d\varphi \\
  & = C \int_{-\pi}^\pi
  \frac{(zcos\varphi, zsin\varphi, a(sin^2\varphi + cos^2\varphi) - rcos\varphi)}{s^3}d\varphi \\
  & = C \int_{-\pi}^\pi
  \frac{(zcos\varphi, zsin\varphi, a - rcos\varphi)}{s^3}d\varphi \\
\end{eqnarray*}

Nu ser vi att $B_x = C\int_{-\pi}^\pi \frac{zcos\varphi}{s^3}d\varphi$ och $B_z
= C\int_{-\pi}^\pi \frac{a - rcos\varphi}{s^3}d\varphi $, men $B_y$ kräver lite mer analys.

\vspace{1pc}

Enligt beräkningen ovan gäller $B_y = C\int_{-\pi}^\pi
\frac{zsin\varphi}{s^3}d\varphi$. Dessutom vet vi att $y = 0$ eftersom vi
arbetar i punkten $p = (r, 0, z)$. Därför måste $sin\varphi = 0$, vilket medför
\begin{eqnarray*}
  B_y & = C\int_{-\pi}^\pi \frac{zsin\varphi}{s^3}d\varphi \\
      & = C\int_{-\pi}^\pi \frac{z \cdot 0}{s^3}d\varphi \\
      & = C \cdot 0 \\
      & = 0.
\end{eqnarray*}

Nu är del 1 nästan klar. Som uppgiftsförfattaren noterar har magnetfältet endast
komponenter i radiell- och z-led. Därför byter vi namn på $B_x$ till $B_r$, då
$y = 0$.

\vspace{1pc}

Del 2 handlar om att nummeriskt lösa diffrentialekvationssystemet 

$$
dr/dv = B_r(r,z),\ dz/dv = B_z(r, z),\ r(0) = r_0,\ z(0) = 0.
$$
Det första problemet vi stöter på är att integralen $\int_{-\pi}^\pi
\frac{zcos\varphi}{s^3}d\varphi$ alltid är lika med 0, eftersom funktionen 
\end{document}
