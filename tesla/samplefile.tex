\documentclass{article}
\usepackage[intlimits]{amsmath}
\usepackage[utf8]{inputenc}

\begin{document}

\centerline{\sc \large 3B.18: Teslatrasslet}
\vspace{.5pc}
\centerline{\sc Snurrig bana i magnetfält}
\vspace{2pc}

Uppgiften kan delas upp i tre delar. Del 1 går ut på att visa att
fältkomponenterna av en vektor 
$$
B(r, 0, z) = \frac{\mu_r\mu_0}{4\pi} \int_{-\pi}^\pi
\frac{\bf{i}(\varphi)\times\bf{s}(\varphi)}{s^3}ad\varphi
$$
med $\mu_0 = 4\pi \cdot 10^{-7}$,\ $\mathbf{i}(\varphi) = I_0(-sin\varphi,
cos\varphi, 0)$,\ $\mathbf{s}(\varphi) = \mathbf{p} - \mathbf{q} = (r -
acos\varphi, -asin\varphi, z)$ och $s = \|p - q\|_2 = \sqrt{(r - acos\varphi)^2
+ a^2sin^2\varphi + z^2}$ blir
$$
B_x = C\int_{-\pi}^\pi \frac{zcos\varphi}{s^3}d\varphi, \
B_y = 0,\
B_z = C\int_{-\pi}^\pi \frac{a - rcos\varphi}{s^3}d\varphi 
$$
där $C = \mu_rI_0a \cdot 10^{-7}$.
% med $\mu_0 = 4\pi \cdot 10^{-7}$, $\bf{i}(\varphi) = I_0(-sin\varphi,
% cos\varphi, 0)$, $\bf{s}(\varphi) = \bf{p} - \bf{q} = (r - acos\varphi,
% -asin\varphi, z)$, 
% $s = \|\bf{p} - \bf{q}\| = \sqrt{(r - acos\varphi)^2 + a^2sin^2\varphi +
% z^2}$,
% $C = \mu_rI_0a \cdot 10^{-7}$.

Det går att lösa enkelt utan nummeriska metoder. 
\begin{eqnarray*}
  B(r, 0, z) & = \frac{\mu_r\mu_0}{4\pi} \int_{-\pi}^\pi
  \frac{\bf{i}(\varphi)\times\bf{s}(\varphi)}{s^3}ad\varphi \\
  & = \frac{a\mu_r\mu_0}{4\pi} \int_{-\pi}^\pi
  \frac{\bf{i}(\varphi)\times\bf{s}(\varphi)}{s^3}d\varphi \\
  & = \frac{4\pi\mu_ra \cdot 10^{-7}}{4\pi} \int_{-\pi}^\pi 
  \frac{\bf{i}(\varphi)\times\bf{s}(\varphi)}{s^3}d\varphi \\
  & = \mu_ra \cdot 10^{-7} \int_{-\pi}^\pi
  \frac{\bf{i}(\varphi)\times\bf{s}(\varphi)}{s^3}d\varphi \\
  & = \mu_ra \cdot 10^{-7} \int_{-\pi}^\pi
  \frac{I_0(-sin\varphi, cos\varphi, 0)\times(r - acos\varphi, -asin\varphi,
z)}{s^3}d\varphi \\
  & = \mu_rI_0a \cdot 10^{-7} \int_{-\pi}^\pi
  \frac{(-sin\varphi, cos\varphi, 0)\times(r - acos\varphi, -asin\varphi,
z)}{s^3}d\varphi \\
  & = C \int_{-\pi}^\pi
  \frac{(-sin\varphi, cos\varphi, 0)\times(r - acos\varphi, -asin\varphi,
z)}{s^3}d\varphi \\
  & = C \int_{-\pi}^\pi
  \frac{(-sin\varphi, cos\varphi, 0)\times(r - acos\varphi, -asin\varphi,
z)}{s^3}d\varphi \\
\end{eqnarray*}

Vi utvecklar kryssprodukten
$
  \mathbf{i}(\varphi) \times \mathbf{s}(\varphi) = (zcos\varphi - 0, 0 - zsin\varphi,
  asin^2\varphi - (rcos\varphi -acos^2\varphi)) = (zcos\varphi, -zsin\varphi,
  a(sin^2\varphi + cos^2\varphi) - rcos\varphi) = (zcos\varphi, zsin\varphi, a -
  rcos\varphi)
$. 
Nu ser vi att komponenterna $B_x = zcos\varphi$ och $B_z = a - rcos\varphi$
men $B_y = 0 \neq zsin\varphi$.
\end{document}
